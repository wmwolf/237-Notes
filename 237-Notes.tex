\documentclass[10pt]{article}

%%%%%%%%%%%%%%%%%%%%%
% Package Inclusion %
%%%%%%%%%%%%%%%%%%%%%
\usepackage{geometry,amsmath,amsthm,mathrsfs,amssymb,graphicx,bm,hyperref,url}

%%%%%%%%%%%%%%%%%%%
% Custom Commands %
%%%%%%%%%%%%%%%%%%%
\newcommand{\n}{\noindent}
\newcommand{\norm}[1]{\left|#1\right|}
\newcommand{\avg}[1]{\left<#1\right>}

%%%%%%%%%%%%%%%%%%%%%%%%%%
% Title Page Information %
%%%%%%%%%%%%%%%%%%%%%%%%%%

\title{Notes for PHYS 237: Galactic Dynamics}
\author{Bill Wolf}
\date{\today}

\begin{document}

\vfill\maketitle\vfill \newpage

\tableofcontents \newpage

%%%%%%%%%%%%%%%%%%%%%%
% September 23, 2011 %
%%%%%%%%%%%%%%%%%%%%%%

\section{Introduction}
	\emph{September 23, 2011}\\
	
	\n The only equation we need is 
	\begin{equation} \label{newton1} \frac{d^2\mathbf{x}}{dt^2}=\sum_{j\neq i}^N Gm_j\frac{\mathbf{x}_j-\mathbf{x}_i}{\norm{\mathbf{x}_j-\mathbf{x}_i}^3}\end{equation}
	which describes the dynamics of a many-particle system.
	\subsection{Analogues}
	Many fields are very similar to the study of galactic dynamics. Early astronomers used these tools to study \textbf{celestial mechanics}, using the theory of gravity to predict the motions of the stars and planets. Additionally, \textbf{statistical mechanics} has close ties with galactic dynamics since in both regimes, we consider the collective behavior of many particles. However, in statistical mechanics, the study is often dominated by short-range forces, whereas in galactic dynamics, short-range forces are (roughly) just as important as long-range forces. Finally, one may draw from the study of \textbf{plasma physics}, which shares the use of a $1/r^2$ law, though it is complicated by the use of positive and negative charges, ushering in the machinery of Maxwell's equations.
	%%%%%
	\subsection{Distribution Functions}
	Quite often, we start with a distribution function that describes how particles are situated in phase space. That is, we describe an ensemble of orbits. In such a distribution function, particles are specified uniquely by position $\mathbf{x}$ and velocity $\mathbf{v}$. We define a \textbf{distribution function} $f(\mathbf{x},\mathbf{v},t)$ as
	\begin{center}
		$f(\mathbf{x},\mathbf{v},t)\,d^3x\,d^3v \equiv$ probability that a particle lies in $\mathbf{x}+d\mathbf{x},\ \mathbf{v}+d\mathbf{v}$
	\end{center}
	Since it's a probability distribution, it must be normalized:
	\begin{equation}\int d^3x\,d^3v\,f(\mathbf{x},\mathbf{v},t)=1\label{normalization1}\end{equation}
	For most systems, this distribution function can only be solved accurately by a computer. Approximations are useful, though, for developing analytic insights into the dynamics of a system.
	%%%%%
	\subsection{Collisions}
	In galactic dynamics, we don't mean that particles necessarily ''collide''. Instead we mean particles come into close causal contact.\\
	
	\n Many physical processes have various degrees to which they are ``collisional''. Fluids, for example, are highly collisional, meaning that the mean free path of any given particle is much smaller than all other relevant length scales. Such a system is described well by Maxwell-Boltzmann statistics (in the non-relativistic, non-degenerate limit). Such an approach does not work for self-gravitating system. Collisional systems are lumpy systems where two-body encounters are important.\\
	
	\n Open star clusters ($N_{\mathrm{stars}}\sim 100$) and planetary systems are good examples of collisional systems. Typically they are few particle systems (contradictory to the fluid dynamics example).\\
	
	\n The other limit is the collisionless systems, or smooth systems. In such a case, there are many particles, so that their ``delta functions'' can be approximated as a smooth distribution (i.e. individual particles less important, large scale structure more important). Here, the underlying gravitational potential can be treated as smooth, and we can use analytic theory to gain insights. Galaxies lie in this regime. For instance, the Milky Way has on the order of $10^11$ stars.\\
	
	\n In the intermediate regime lie things like globular clusters ($N_{\mathrm{stars}}\sim 10^3-10^5$)
	
	\subsection{The Virial Theorem}
	The Virial Theorem comes up in many contexts, including astrophysics and statistical mechanics. The exact statement of the theorem (for our use) is 
	\begin{equation} \label {virial1} 2\avg{T} + \avg{V} = 0\end{equation}
	where $T$ and $V$ are the kinetic and potential energies of a system, respectively. One might notice that it's important that $T<-V$ in order that a system might be bound. This theorem is extremely useful (assuming a system has come to virial equilibrium). For instance, the velocities in a galaxy can be determined via doppler broadening of the spectra. From this, we find the kinetic energy, which then gives us a potential energy via \eqref{virial1}, giving us the mass!\\
	
	\n\begin{proof}
		The potential for a gravitationally-bound system is
		$$V=-\frac{1}{2}\sum_{i\neq j}\frac{Gm_im_j}{\norm{\mathbf{x}_{ij}}}$$
		To order of magnitude, 
		$$V \sim \frac{GM^2}{r_h}$$
		where $r_h$ is some characteristic size. The kinetic energy is
		$$T = \frac{1}{2}\sum_j m_j\norm{\dot{\mathbf{x}}_j}^2$$
		Consider a system of point particles with positions $\mathbf{r}_i$ and momenta $\mathbf{p}_i$. Consider the quantity
		$$G=\sum_{i}\mathbf{p}_i\cdot\mathbf{x}_i$$
		Differentiating with respect to time gives
		$$\dot{G} = \sum_i \dot{\mathbf{r}}_i\cdot \mathbf{p}_i+\sum_i\dot{\mathbf{p}}_i\cdot\mathbf{r}_i=\sum_i\mathbf{p}_i\cdot\frac{\mathbf{p}_i}{m}+\sum_i\mathbf{F}_i\cdot\mathbf{r}_i = 2T+\sum_i\mathbf{F}_i\cdot\mathbf{r}_i$$
		This quantity is called the virial (no clue why). If we take a time average,
		$$\avg{\dot{G}} = \frac{1}{\tau}\int_0^\tau\frac{dG}{dt'}\,dt' = \frac{1}{\tau}\left[G(\tau)-G(0)\right]\to 0\ \textrm{as}\ \tau\to \infty$$
		Consider $\sum_i\mathbf{F}_i\cdot\mathbf{r}_i$ for gravitational force
		$$\mathbf{F}_i = \sum_{j\neq i}\frac{Gm_im_j}{\norm{\mathbf{r}_{ij}}^2}$$
		So we have
		$$\sum_i \mathbf{F}_i\cdot\mathbf{r}_i=\sum_{i,j}\frac{Gm_im_j\mathbf{R}_j\cdot(\mathbf{r}_j-\mathbf{r}_i)}{\norm{\mathbf{r}_j-\mathbf{r}_i}^3}$$		
	\end{proof}
	
	\n\emph{September 26, 2011}
	\section{Potential Theory}
	In this class, we will only consider spherical and disk potentials since most self-gravitating objects tend to be spherical due to the radial nature of the gravitational force, but conservation of angular momentum can cause disk-like structure.
	For a distribution of point particles (collisionless limit), we will approximate the density has a smooth function. We will solve for the potential in two different ways. First, via Poisson's equation:
	\begin{equation}\label{poisson} \nabla^2\Phi = 4\pi G\rho\end{equation}
	and also via Gauss' Law:
	\begin{equation}\label{gauss1} \Phi(\mathbf{x})=-G\int\frac{\rho(\mathbf{x}')}{\norm{\mathbf{x}-\mathbf{x}'}}d^3x'\end{equation}
	or its more useful form
	\begin{equation} \label{gauss2} \int (\nabla\Phi)\cdot d\mathbf{S}=4\pi G M_{\mathrm{enc}}\end{equation}
	\subsection{Spherical Potentials}
	For a spherical shell of mass, the potential measured outside of the shell is the same as that of a point particle in the center:
	\begin{equation}\label{shellpot1} \Phi_{\mathrm{ext}}(r) = -\frac{GM_{\mathrm{shell}}}{r}\end{equation}
	Alternatively,
	\begin{align}
		\int (\nabla\cdot\Phi)d\S = 4\pi r^2 F &= 4\pi G M_{\mathrm{encl}}\nonumber\\
		&\Rightarrow F=-\frac{GM_{\mathrm{encl}}}{r^2}\label{shellpot2}
	\end{align}
	For a spherical distribution of mass, we may again use Gauss' law. However, when just considering an interior shell, the exterior shells are still important since they add on a constant potential
	\paragraph{Interior Shell:} $r'<r$
	$$\delta \Phi = -\frac{GM_{\mathrm{shell}}}{r}=-\frac{G\rho(r')4\pi{r'}^2dr'}{r}$$
	\paragraph{Exterior Shell:} $\delta \Phi$ is constant. Let us choose $\delta\Phi$ so that it is continuous
	$$\delta\Phi = -\frac{GM_{\mathrm{shell}}}{r'}=-\frac{G\rho(r') 4\pi {r'}^2dr'}{r'}$$
	Then the overall potential is
	\begin{equation} \label{potential1} \Phi(r)=\int_{\mathrm{int}}+\int_{\mathrm{ext}} = -G\left[\frac{1}{r}\int 4\pi{r'}^2 \rho(r')\,dr'+\int _r^\infty 4\pi \rho(r')r'\,dr'\right]\end{equation}
	\subsubsection{Measurements}
	We can't actually measure the density profile of most astrophysical objects, but we \emph{can} measure velocities via the Doppler Effect. We get the \textbf{circular velocity} and deduce a gravitational force:
	\begin{equation}\label{circular}F=\frac{mv_c^2}{r}=\frac{GMm}{r^2}\Rightarrow v_c^2=\frac{GM(r)}{r}\end{equation}
	which gives us a direct measurement of the mass profile.\\
	
	\n The \textbf{escape velocity} is also an important concept. Essentially, it is the velocity a particle must have to escape to $r\to\infty$. That is,
	\begin{equation}\label{escape1} \frac{1}{2}mv_{\mathrm{esc}}^2-\frac{GMm}{r}=0\Rightarrow v_{\mathrm{esc}}^2=\frac{2GM}{r}={2v_{c}}^2\end{equation}
	\subsubsection{Examples of Spherically Symmetric Potentials}
	\textbf{Homogeneous Sphere} In this scenario, the density is constant ($\rho(r)=\mathrm{const}$). The gravitational force is given by
	\begin{equation}\label{homogeneous1} f=-\frac{GM}{r^2}=-\frac{G(4/3 \pi\rho r^3)}{r^2}\end{equation}
	which yields the ODE
	\begin{equation}\label{homogeneous2} \ddot{r}=-\left(\frac{4G\pi \rho}{3}\right)r\end{equation}
	with obvious harmonic solution. Thus, the period of \emph{every} orbit is 
	\begin{equation}\label{homogeneous3} P=\frac{2\pi}{\omega}=2\pi\sqrt{\frac{3}{4\pi G\rho}}\propto \frac{1}{\sqrt{G\rho}}\end{equation}
	Note that we have shown the dynamical time scale for the systems gravitational evolution, $\tau=1/\sqrt{G\rho}$.\\
	
	\n Now we shift our attention to the potential of a homogeneous sphere with radius $b$. For $r>b$, then we have the obvious
	$$\Phi(r)=-\frac{GM}{r}$$
	For $r<b$, we have to specify the constant potential to ensure continuity. Noting that $M_{\mathrm{tot}} = 4/3 \pi\rho b^3$,
	\begin{align*}
	\Phi_<&=-G\left[\frac{1}{r}\frac{4}{3}\pi\rho r^3 + \int_r^b 4\pi g r'\,dr'\right]\\
	      &=-\frac{G}{r} M_{\mathrm{tot}}\left(\frac{r^3}{b^3}\right)+2\pi G(b^2-r^2)\\
	      &=-\frac{GM_{\mathrm{tot}}}{b}\left[\frac{r^2}{b^2}+\frac{3}{2}\left(1-\frac{r^2}{b^2}\right)\right]\\
	      &=-\frac{GM_{\mathrm{tot}}}{b}\left[\frac{3}{2}-\frac{r^2}{2b^2}\right]
	\end{align*}
	\paragraph{Singular Isothermal Sphere} The singular isothermal sphere  is one whose rotation curve increases in a Keplarian fashion up to a certain point and then becomes constant with increasing $r$. That is, after a certain point, the rotational velocity becomes independent with radius. Evidence of this is one of the top indicators for the presence of dark matter. The word ``isothermal'' is present since physicists often relate the temperature with the velocity of particles (not a good name). The word ``singular'' is present since the density is seen to diverge at the center.\\
	
	\n As it turns out, computational simulations for dark matter halos find $\rho\propto r^{-1}$ for $r\to 0$ and $\rho\propto r^{-3}$ as $r\to \infty$.\\
	
	\n We wish to know how to extract the mass profile for this velocity distribution. Turns out that
	$$M(r)\propto \int \rho r^2\,dr\propto r$$
	which is bad, since the density diverges towards the center. Often, we truncate the density at a certain ``Virial radius''. Then the mass is set to vary as
	$$M=M_0\left(\frac{r}{r_0}\right)$$
	In the outer regime,
	$$v_c^2=\frac{GM}{r}=\mathrm{const}$$
	We're really interested in the potential between shells:
	\begin{equation}\label{singular1} \Phi(r)-\Phi(r_0)=G\int_{r_0}^{r}dr'\,\frac{GM(r')}{{r'}^2} = \frac{GM_0}{r_0}\ln\left(\frac{r}{r_0}\right)=v_c^2\ln\left(\frac{r}{r_0}\right)\end{equation}
	
	
	
	
	
	
\end{document}